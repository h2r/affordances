\documentstyle[proceed]{article} 

\title{Planning with Affordances}

\begin{document}

\maketitle

\begin{abstract}
We introduce a novel approach to planning that combines the concept of affordances \cite{Gibson} with a standard planning algorithm, Value Iteration. Classical planning algorithms suffer from combinatoric state-space explosions \cite{Norvig} that cripple their effectiveness. Thus, in order to fully realize the potential of planning algorithms, we have sought to make these previously difficult problems more tractable; through the use of our affordance framework we seek to ``guide" the agent as it plans, significantly reducing the size of exponential state-spaces. To accomplish this, we propose a planning algorithm that prunes the state action space using affordances. Evaluation is performed in the Minecraft domain on several path planning tasks - we demonstrate a significant increase in speed and reduction in state-space exploration across 5 different path planning tasks.
\end{abstract}

\section{INTRODUCTION}

\section{BACKGROUND}

\subsection{SUBGOALS}

\subsection{OPTIONS}

\subsection{MACRO-ACTIONS}

\subsection{OO-MDP}

\section{AFFORDANCE FORMALISM}

\section{EXPERIMENTS}

\subsection{BASELINES}

\section{RESULTS}

\end{document}